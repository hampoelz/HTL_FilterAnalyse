\documentclass[a4paper]{hitec}
\settextfraction{0.95}      % reduce left margin

\usepackage{styles/main}
\usepackage{styles/custom}

\htltitle{Bewertung von Filterschaltungen}
\confidential{\textbf{Elektrotechnik}}

\author{Rene Hampölz}
\company{HTBLA Weiz}
\date{2022}

\begin{document}

\maketitletoc
\clearpage

\section{Einführung}

Als messtechnische Maßnahme sollen Störfrequenzen unterhalb einer definierten Grenzfrequenz von \qty{30}{\hertz} stark gedämpft werden. Es ist bekannt, dass zwischen \qty{50}{\hertz} und \qty{10}{\kilo\hertz} die Frequenzanteile des Nutzsignals auftreten, welche möglichst ungedämpft übertragenwerden sollen. Für die beschriebene Funktion sollen verschiedene Filter ausgelegt, simuliert und analysiert werden. Die Eigenschaften der Filter sind zu vergleichen sowie deren Eignung in Bezug auf die Aufgabenstellung zu diskutieren.

\begin{sagesilent}
    f_g = 30
    R = 100

    s = var('s')
\end{sagesilent}

Angaben: $f_{g} = \SIvar{f_g}{\hertz}$, $R = \SIvar{R}{\ohm}$

\section{Entwurf eines RL-Filter erster Ordnung}

\begin{figure}[H]
    \centering
    \begin{circuitikz}
        \coordinate (in+) at (0,0);
        \coordinate (in-) at (0,-2);

        \draw
        (in+) to[R=$R$, a=$\SIvar{R}{\ohm}$, o-] ++(4,0) coordinate (aux1)
        to[short,-o] ++(2,0) coordinate (out+)

        (aux1) to[cute inductor=$L$,*-*] (aux1 |- in-)

        (in-) to[short,o-o] (in- -| out+) coordinate (out-)

        (in+) to[open,name=inV,v] (in-)
        (out+) to[open,name=outV,v^] (out-)
        ;

        \voltage{inV}{$U_e$}
        \voltage{outV}{$U_a$}
    \end{circuitikz}
\end{figure}

\begin{equation*}
    F_{(jw)}
    = \frac{\underline{U_{a}}}{\underline{U_{e}}}
    = \frac{jwL}{jwL+R} \cdot \frac{\frac{1}{jwL}}{\frac{1}{jwL}}
    = \frac{1}{1+\frac{R}{jwL}} = \frac{1}{\underbrace{1}_{Re} - \underbrace{j \cdot \frac{R}{wL}}}_{Im}
\end{equation*}

\subsection{Betrag und Phase aus der Übertragungsfunktion}

\begin{align*}
    Re &= 1 \\
    Im &= -j \cdot \frac{R}{wL}
\end{align*}

\begin{align*}
    |G_{jw}| & = \frac{1}{\sqrt{1^2 + (\frac{R}{wL})^2}} &
    \varphi &= arctan(-\frac{R}{wL})
\end{align*}

\subsection{Ermittlung der Asymptoten des Bodediagramms}

\begin{equation*}
    w = \infty \Rightarrow |F| = \frac{1}{1 - j \cdot \frac{R}{wL}} = \frac{1}{1 - 0} = 1
\end{equation*}

\begin{equation*}
    w = 0 \Rightarrow |F| = \frac{1}{1 - j \cdot \frac{R}{wL}} = \frac{1}{- \infty} = 0
\end{equation*}

\subsection{Berechnung der Bauteile}

\begin{sagesilent}
    L = R/(2*pi*f_g)
\end{sagesilent}

\begin{align*}
    f_{g} &= \frac{R}{2 \pi L} \\
    \Rightarrow L &= \frac{R}{2 \pi f_{g}} \\
    L &= \frac{\var{R}}{2 \pi \var{f_g}} \\
    L &= \SIvar{L, 3}{\henry}
\end{align*}

\subsection{Übertragungsfunktion}

\begin{sagesilent}
    bode_F1 = TransferFunction((s*L)/(R+s*L), 1, 10000, 1)
    bode_F1_freq_mag = bode_F1.plot_data_freq_mag_bode()
    bode_F1_freq_pha = bode_F1.plot_data_freq_pha_bode()

    bode_F1_plot_mag = list_plot(
        bode_F1_freq_mag,
        axes_labels=["Frequenz in Hz", "Betrag in dB"],
        scale='semilogx',
        figsize=[5.5,2],
        plotjoined=True,
        frame=True,
        color='red'
    )

    bode_F1_plot_pha = list_plot(
        bode_F1_freq_pha,
        axes_labels=["Frequenz in Hz", "Phase in $\degree$"],
        scale='semilogx',
        figsize=[5.5,2],
        plotjoined=True,
        frame=True,
        color='red'
    )
\end{sagesilent}

\begin{figure}[H]
    \centering
    \begin{subfigure}{\textwidth}
        \centering
        \sageplot{bode_F1_plot_mag}
    \end{subfigure}
    \quad
    \begin{subfigure}{\textwidth}
        \centering
        \sageplot{bode_F1_plot_pha}
    \end{subfigure}
    \caption{Amplituden- und Phasengang des RL-Hochpasses}
    \label{fig:F1}
\end{figure}

\begin{sagesilent}
    bode_F1_mag_1 = bode_F1_freq_mag[0][1]
    bode_F1_mag_10 = bode_F1_freq_mag[9][1]
    attenuation_dB_F1 = bode_F1_mag_10 - bode_F1_mag_1
    attenuation_dB_F1 = attenuation_dB_F1
\end{sagesilent}

Die Abschwächung im Frequenzbereich von \qty{1}{\hertz} bis \qty{10}{\hertz} beträgt $\var{attenuation_dB_F1, 0}\,\nicefrac{\unit{dB}}{\text{Dekade}}$.

\begin{sagesilent}
    bode_F1_mag_fg = bode_F1_freq_mag[f_g-1][1]
    bode_F1_pha_fg = bode_F1_freq_pha[f_g-1][1]
\end{sagesilent}

Bei der Grenzfrequenz von \SIvar{f_g}{\hertz} ergibt sich eine Verstärkung von \SIvar{bode_F1_mag_fg, 2}{\decibel} und eine Phasenverschiebung von \SIvar{bode_F1_pha_fg, 0}{\degree}

\subsection{Simulation}

\begin{sagesilent}
    bode_F1_sim_data = LTSpice.plot_data_polar("src/simulations/export/01 RL-Filter.txt")
    bode_F1_sim_data_mag, bode_F1_sim_data_pha = bode_F1_sim_data["V(out)"]

    bode_F1_sim_plot_mag = list_plot(
        bode_F1_sim_data_mag,
        axes_labels=["Frequenz in Hz", "Betrag in dB"],
        scale='semilogx',
        figsize=[5.5,2],
        plotjoined=True,
        frame=True,
        color='darkcyan'
    )

    bode_F1_sim_plot_pha = list_plot(
        bode_F1_sim_data_pha,
        axes_labels=["Frequenz in Hz", "Phase in $\degree$"],
        scale='semilogx',
        figsize=[5.5,2],
        plotjoined=True,
        frame=True,
        color='darkcyan'
    )
\end{sagesilent}

\begin{figure}[H]
    \centering
    \begin{subfigure}{\textwidth}
        \centering
        \sageplot{bode_F1_sim_plot_mag}
    \end{subfigure}
    \quad
    \begin{subfigure}{\textwidth}
        \centering
        \sageplot{bode_F1_sim_plot_pha}
    \end{subfigure}
    \caption{Amplituden- und Phasengang der Simulation des RL-Hochpasses}
    \label{fig:F1_Sim}
\end{figure}
\section{Entwurf eines RLC-Filter zweiter Ordnung}

Tiefere Frequenzen sollen stärker abgeschwächt werden \textit{($\num{-40}\,\nicefrac{\unit{dB}}{\text{Dekade}}$)}.
Jedoch darf, zusätzlich zu den bereits verwendeten Bauelementen, nur eine Kapazität verwendet werden.

\begin{figure}[H]
    \centering
    \begin{circuitikz}
        \coordinate (in+) at (0,0);
        \coordinate (in-) at (0,-2);

        \draw
        (in+) to[R=$R$, a=$\SIvar{R}{\ohm}$, o-] ++(2,0)
        to[C=$C$] ++(2,0) coordinate (aux1)
        to[short,-o] ++(2,0) coordinate (out+)

        (aux1) to[cute inductor=$L$,*-*] (aux1 |- in-)

        (in-) to[short,o-o] (in- -| out+) coordinate (out-)

        (in+) to[open,name=inV,v] (in-)
        (out+) to[open,name=outV,v^] (out-)
        ;

        \voltage{inV}{$U_e$}
        \voltage{outV}{$U_a$}
    \end{circuitikz}
\end{figure}

\begin{equation*}
    F_{(jw)}
    = \frac{\underline{U_{a}}}{\underline{U_{e}}}
    = \frac{jwL}{jwL+R+\frac{1}{jwC}} \cdot \frac{\frac{1}{jwL}}{\frac{1}{jwL}}
    = \frac{1}{1+\frac{\frac{1}{jwC}}{jwL}+\frac{R}{jwL}}
    = \frac{1}{\underbrace{1-\frac{1}{w^2LC}}_{Re} + \underbrace{\frac{R}{jwL}}_{Im}}
\end{equation*}

\subsection{Betrag und Phase aus der Übertragungsfunktion}

\begin{align*}
    Re &= 1 - \frac{1}{w^2LC} \\
    Im &= \frac{R}{jwL}
\end{align*}

\begin{align*}
    |G_{jw}| & = \frac{1}{\sqrt{(1 - \frac{1}{w^2LC})^2 + (\frac{R}{wL})^2}} &
    \varphi &= arctan(\frac{1 - \frac{1}{w^2LC}}{\frac{R}{wL}})
\end{align*}

\subsection{Berechnung der Bauteile}

\begin{sagesilent}
    L = R/(2*pi*f_g)
    C = var('C')

    X_L = 2*pi*f_g*L
    X_C = 1/(2*pi*f_g*C)

    C = solve([X_L == X_C], C)[0].rhs()
\end{sagesilent}

\begin{align*}
    f_{g} &= \frac{R}{2 \pi L} \\
    \Rightarrow L &= \frac{R}{2 \pi f_{g}} \\
    L &= \frac{\var{R}}{2 \pi \var{f_g}} \\
    L &= \SIvar{L, 3}{\henry}
\end{align*}

\begin{align*}
    X_{L} &= 2 \pi f_{g} L \\
    X_{C} &= \frac{1}{2 \pi f_{g} C} \\
    X_{L} &= X_{C} \\
    \Rightarrow C &= \SIvar{C*10^6, 1}{\micro\farad}
\end{align*}

\subsection{Übertragungsfunktion}

\begin{sagesilent}
    bode_F2 = TransferFunction((s*L)/(s*L+R+1/(s*C)), 1, 10000, 1)
    bode_F2_freq_mag = bode_F2.plot_data_freq_mag_bode()
    bode_F2_freq_pha = bode_F2.plot_data_freq_pha_bode()

    bode_F2_plot_mag = list_plot(
        bode_F2_freq_mag,
        axes_labels=["Frequenz in Hz", "Betrag in dB"],
        legend_label="RLC-Filter",
        scale='semilogx',
        figsize=[5.5,2],
        plotjoined=True,
        frame=True,
        color='red'
    )

    bode_F2_plot_mag += list_plot(
        bode_F1_freq_mag,
        legend_label="RL-Filter",
        plotjoined=True,
        linestyle=':',
        color='blue'
    )

    bode_F2_plot_pha = list_plot(
        bode_F2_freq_pha,
        axes_labels=["Frequenz in Hz", "Phase in $\degree$"],
        legend_label="RLC-Filter",
        scale='semilogx',
        figsize=[5.5,2],
        plotjoined=True,
        frame=True,
        color='red'
    )

    bode_F2_plot_pha += list_plot(
        bode_F1_freq_pha,
        legend_label="RL-Filter",
        plotjoined=True,
        linestyle=':',
        color='blue'
    )
\end{sagesilent}

\begin{figure}[H]
    \centering
    \begin{subfigure}{\textwidth}
        \centering
        \sageplot{bode_F2_plot_mag}
    \end{subfigure}
    \quad
    \begin{subfigure}{\textwidth}
        \centering
        \sageplot{bode_F2_plot_pha}
    \end{subfigure}
    \caption{Amplituden- und Phasengang des RLC-Hochpasses und Vergleich mit dem RL-Hochpass (Abb. \ref{fig:F1}) der vorherigen Übung.}
    \label{fig:F2}
\end{figure}

\subsection{Übertragungsfunktion mit $C_1=0.1mF$ und $C_2=0.01mF$}

\begin{sagesilent}
    C_1 = 0.1*10^-3
    bode_F2_C1 = TransferFunction((s*L)/(s*L+R+1/(s*C_1)), 1, 10000, 1)
    bode_F2_C1_freq_mag = bode_F2_C1.plot_data_freq_mag_bode()
    bode_F2_C1_freq_pha = bode_F2_C1.plot_data_freq_pha_bode()

    C_2 = 0.01*10^-3
    bode_F2_C2 = TransferFunction((s*L)/(s*L+R+1/(s*C_2)), 1, 10000, 1)
    bode_F2_C2_freq_mag = bode_F2_C2.plot_data_freq_mag_bode()
    bode_F2_C2_freq_pha = bode_F2_C2.plot_data_freq_pha_bode()
    
    bode_F2_C12_plot_mag = list_plot(
        bode_F2_C1_freq_mag,
        axes_labels=["Frequenz in Hz", "Betrag in dB"],
        legend_label="$C_1$",
        scale='semilogx',
        figsize=[5.5,2],
        plotjoined=True,
        frame=True,
        color='green'
    )

    bode_F2_C12_plot_mag += list_plot(
        bode_F2_C2_freq_mag,
        legend_label='$C_2$',
        plotjoined=True,
        color='purple'
    )

    bode_F2_C12_plot_pha = list_plot(
        bode_F2_C1_freq_pha,
        axes_labels=["Frequenz in Hz", "Phase in $\degree$"],
        legend_label="$C_1$",
        scale='semilogx',
        figsize=[5.5,2],
        plotjoined=True,
        frame=True,
        color='green'
    )

    bode_F2_C12_plot_pha += list_plot(
        bode_F2_C2_freq_pha,
        legend_label="$C_2$",
        plotjoined=True,
        color='purple'
    )
\end{sagesilent}

\begin{figure}[H]
    \centering
    \begin{subfigure}{\textwidth}
        \centering
        \sageplot{bode_F2_C12_plot_mag}
    \end{subfigure}
    \quad
    \begin{subfigure}{\textwidth}
        \centering
        \sageplot{bode_F2_C12_plot_pha}
    \end{subfigure}
    \caption{Amplituden- und Phasengang des RLC-Hochpasses.}
    \label{fig:F2_C12}
\end{figure}

\subsection{Analyse}

Die Kapazität $C$ hat einen direkten Einfluss auf die Übertragungsbandbreite des RLC-Hochpassfilters und bestimmt in gewisser Weise, wie gut der Filter bestimmte Frequenzen durchlässt oder abfiltert.
Bei größeren Kapazitäten, lässt der der Hochpassfilter eine größere Bandbreite an höheren Frequenzen durch.

Bei der Kapazität $C_2$ (\SIvar{C_1*10^3, 1}{\milli\farad}) stimmt die Grenzfrequenz $f_g$ von \SIvar{f_g}{\hertz} nicht mehr überein und ist daher nicht geeignet, deshalb wird für weitere Analysen die Kapazität $C_2$ (\SIvar{C_2*10^3, 2}{\milli\farad}) verwendet.

\begin{sagesilent}
    bode_F2_mag_1 = bode_F2_C1_freq_mag[0][1]
    bode_F2_mag_10 = bode_F2_C1_freq_mag[9][1]

    attenuation_dB_F2 = bode_F2_mag_10 - bode_F2_mag_1
    attenuation_dB_F2 = attenuation_dB_F2
\end{sagesilent}

Die Abschwächung im Frequenzbereich von \qty{1}{\hertz} bis \qty{10}{\hertz} beträgt somit $\var{attenuation_dB_F2, 0}\,\nicefrac{\unit{dB}}{\text{Dekade}}$.

\begin{sagesilent}
    bode_F2_mag_fg = bode_F2_C1_freq_mag[f_g-1][1]
    bode_F2_pha_fg = bode_F2_C1_freq_pha[f_g-1][1]
\end{sagesilent}

Bei der Grenzfrequenz von \SIvar{f_g}{\hertz} ergibt sich dabei eine Verstärkung von \SIvar{bode_F2_mag_fg, 2}{\decibel} und eine Phasenverschiebung von \SIvar{bode_F2_pha_fg, 0}{\degree}

\subsection{Simulation}

\begin{sagesilent}
    bode_F2_sim_data = LTSpice.plot_data_polar("src/simulations/export/02 RLC-Filter.txt")
    bode_F2_sim_data_mag, bode_F1_sim_data_pha = bode_F2_sim_data["V(out)"]

    bode_F2_sim_plot_mag = list_plot(
        bode_F2_sim_data_mag,
        axes_labels=["Frequenz in Hz", "Betrag in dB"],
        scale='semilogx',
        figsize=[5.5,2],
        plotjoined=True,
        frame=True,
        color='darkcyan'
    )

    bode_F2_sim_plot_pha = list_plot(
        bode_F1_sim_data_pha,
        axes_labels=["Frequenz in Hz", "Phase in $\degree$"],
        scale='semilogx',
        figsize=[5.5,2],
        plotjoined=True,
        frame=True,
        color='darkcyan'
    )
\end{sagesilent}

\begin{figure}[H]
    \centering
    \begin{subfigure}{\textwidth}
        \centering
        \sageplot{bode_F2_sim_plot_mag}
    \end{subfigure}
    \quad
    \begin{subfigure}{\textwidth}
        \centering
        \sageplot{bode_F2_sim_plot_pha}
    \end{subfigure}
    \caption{Amplituden- und Phasengang der Simulation des RLC-Hochpasses}
    \label{fig:F2_Sim}
\end{figure}
\section{Weitere Analyse von Filter 1 und 2}

\subsection{Sprungantwort}

\begin{sagesilent}
    bode_F1_step = bode_F1.plot_data_step()
    bode_F2_step = bode_F2_C1.plot_data_step()
    
    bode_F12_plot_step = list_plot(
        bode_F1_step,
        axes_labels=["Zeit in s", "Amplitude"],
        legend_label="RL-Filter",
        figsize=[5.5,2],
        plotjoined=True,
        frame=True,
        color='green'
    )

    bode_F12_plot_step += list_plot(
        bode_F2_step,
        legend_label="RLC-Filter",
        plotjoined=True,
        color='blue'
    )
\end{sagesilent}

\begin{figure}[H]
    \centering
    \sageplot{bode_F12_plot_step}
    \caption{Sprungantwort des RL- und RLC-Hochpasses.}
    \label{fig:F12_Step}
\end{figure}

Die Sprungantwort beschreibt, wie sich das Ausgangssignal ändert, wenn ein plötzlicher Sprung im Eingangssignal auftritt.
Dieser Sprung kann zum Beispiel durch einen schnellen Übergang von einem niedrigen auf eine hohes Eingangssignal verursacht werden.

\subsection{Gruppenlaufzeit}

Die Gruppenlaufzeit kann aus dem Verlauf der Phase berechnet werden. Sie ist die negative Ableitung des Phasenverlaufs nach der Kreisfrequenz. Für die Gruppenlaufzeit $G$ ergibt sich also:

\begin{equation*}
    G = \frac{d \varphi}{d w}
\end{equation*}

Um eine Ableitung berechnen zu können muss der allgemeine Phasengang $\varphi$ vorliegen. Es folgt dann die Ableitung nach $w$:

\vspace{1em}

\begin{sagesilent}
    # Group-Delay LaTeX Formular:
    # var('w R L C')
    # phi = arctan(-(R/(w*L)))
    # phi = arctan((1-(1/(w^2*L*C)))/(R/(w*L)))
    # latex(derivative(phi, w).full_simplify())

    w_f50 = 2*pi*50
    bode_F1_G_f50 = bode_F1.get_group_delay(w_f50)
    bode_F2_G_f50 = bode_F2.get_group_delay(w_f50)
\end{sagesilent}

\textbf{RL-Filter:}

\begin{align*}
    \varphi &= arctan(-\frac{R}{wL}) \\
    G &= \frac{R l}{l^{2} w^{2} + R^{2}} \\
    G &= \SIvar{bode_F1_G_f50 * 10^3, 3}{\milli\second} \tag*{bei $f=\qty{50}{\hertz}$}
\end{align*}

\textbf{RLC-Filter:}

\begin{align*}
    \varphi &= arctan(\frac{1 - \frac{1}{w^2LC}}{\frac{R}{wL}}) \\
    G &= \frac{C^{2} L R w^{2} + C R}{C^{2} L^{2} w^{4} + {\left(C^{2} R^{2} - 2 \, C L\right)} w^{2} + 1} \\
    G &= \SIvar{bode_F2_G_f50 * 10^3, 3}{\milli\second} \tag*{bei $f=\qty{50}{\hertz}$}
\end{align*}

\vspace{1em}

\begin{sagesilent}
    bode_F1_G = bode_F1.plot_data_freq_gd(f_stop=250)
    bode_F2_G = bode_F2.plot_data_freq_gd(f_stop=250)

    bode_F12_plot_G = list_plot(
        bode_F1_G,
        axes_labels=["Frequenz in Hz", "Zeit in s"],
        legend_label="RL-Filter",
        figsize=[5.5,2],
        plotjoined=True,
        frame=True,
        color='green'
    )

    bode_F12_plot_G += list_plot(
        bode_F2_G,
        legend_label="RLC-Filter",
        plotjoined=True,
        color='blue'
    )
\end{sagesilent}

\begin{figure}[H]
    \centering
    \sageplot{bode_F12_plot_G}
    \caption{Gruppenlaufzeit des RL- und RLC-Hochpasses.}
    \label{fig:F12_G}
\end{figure}

Die Gruppenlaufzeit ist die Verzögerung eines Signals, beim Durchgang durch einen Filter.
Sie beschreibt, wie lange eine Signalwelle durch einen Filter benötigt.

\subsection{Simulation mit Belastung $R_L=100\Omega$}

\begin{sagesilent}
    bode_F12_sim_data = LTSpice.plot_data_polar("src/simulations/export/03 RL- RLC-Filter Belastung.txt")

    bode_F12_sim_data_F1_mag, bode_F12_sim_data_F1_pha = bode_F12_sim_data["V(rl_out)"]
    bode_F12_sim_data_F2_mag, bode_F12_sim_data_F2_pha = bode_F12_sim_data["V(rlc_out)"]

    bode_F12_sim_plot_mag = list_plot(
        bode_F12_sim_data_F1_mag,
        axes_labels=["Frequenz in Hz", "Betrag in dB"],
        legend_label="RL-Filter",
        scale='semilogx',
        figsize=[5.5,2],
        plotjoined=True,
        frame=True,
        color='darkcyan'
    )

    bode_F12_sim_plot_mag += list_plot(
        bode_F12_sim_data_F2_mag,
        legend_label="RLC-Filter",
        plotjoined=True,
        color='darkslateblue'
    )

    bode_F12_sim_plot_pha = list_plot(
        bode_F12_sim_data_F1_pha,
        axes_labels=["Frequenz in Hz", "Phase in $\degree$"],
        legend_label="RL-Filter",
        scale='semilogx',
        figsize=[5.5,2],
        plotjoined=True,
        frame=True,
        color='darkcyan'
    )

    bode_F12_sim_plot_pha += list_plot(
        bode_F12_sim_data_F2_pha,
        legend_label="RLC-Filter",
        plotjoined=True,
        color='darkslateblue'
    )
\end{sagesilent}

\begin{figure}[H]
    \centering
    \begin{subfigure}{\textwidth}
        \centering
        \sageplot{bode_F12_sim_plot_mag}
    \end{subfigure}
    \quad
    \begin{subfigure}{\textwidth}
        \centering
        \sageplot{bode_F12_sim_plot_pha}
    \end{subfigure}
    \caption{Amplituden- und Phasengang der Simulation des RL- und RLC-Hochpasses unter Belastung}
    \label{fig:F12_Sim}
\end{figure}

Eine ohmsche Belastung beeinflusst die Koeffizienten, welche für die Berechnung der Übertragungsfunktion verwendet werden.
Dies führt dazu, dass die Grenzfrequenz des Filters sowie dessen Amplituden-Frequenzgang verändert werden.  
\section{Entwurf eines aktiven RLC-Filter zweiter Ordnung}

Im Zusammenhang mit den ermittelten Bauteilen vom RLC-Filter, soll ein aktiver Filter mit einem Operationsverstärker OP747 entworfen werden.
Der OPV soll als Spannungsfolger im Dual-Supply Mode eingesetzt werden.

\begin{figure}[H]
    \centering
    \begin{circuitikz}
        \coordinate (in+) at (0,0);
        \coordinate (in-) at (0,-3);

        \draw
        (in+) to[R=$R$, a=$\SIvar{R}{\ohm}$, o-] ++(2,0)
        to[C=$C$, a=$\SIvar{C_1*10^3, 1}{\milli\farad}$] ++(2,0) coordinate (aux1)
        -- ++(2,0) node[op amp,anchor=-](OPV){OPV}

        (OPV.out) to[short, -o] ++(1,0) coordinate (out+)

        (OPV.+) -- ++(0, -1) coordinate(opv-aux1)
        -- (opv-aux1 -| OPV.out) to[short,-*] (OPV.out)
        
        (aux1) to[cute inductor=$L$, a=$\SIvar{L, 3}{\henry}$, *-*] (aux1 |- in-)

        (in-) to[short,o-o] (in- -| out+) coordinate (out-)

        (in+) to[open,name=inV,v] (in-)
        (out+) to[open,name=outV,v^] (out-)
        ;

        \voltage{inV}{$U_e$}
        \voltage{outV}{$U_a$}
    \end{circuitikz}
\end{figure}

\begin{equation*}
    F_{(jw)}
    = \frac{\underline{U_{a}}}{\underline{U_{e}}}
    = \frac{jwL}{jwL+R+\frac{1}{jwC}} \cdot \frac{\frac{1}{jwL}}{\frac{1}{jwL}}
    = \frac{1}{1+\frac{\frac{1}{jwC}}{jwL}+\frac{R}{jwL}}
    = \frac{1}{\underbrace{1-\frac{1}{w^2LC}}_{Re} + \underbrace{\frac{R}{jwL}}_{Im}}
\end{equation*}

\subsection{Betrag und Phase aus der Übertragungsfunktion}

\begin{align*}
    Re &= 1 - \frac{1}{w^2LC} \\
    Im &= \frac{R}{jwL}
\end{align*}

\begin{align*}
    |G_{jw}| & = \frac{1}{\sqrt{(1 - \frac{1}{w^2LC})^2 + (\frac{R}{wL})^2}} &
    \varphi &= arctan(\frac{1 - \frac{1}{w^2LC}}{\frac{R}{wL}})
\end{align*}

\subsection{Simulation}

\begin{sagesilent}
    bode_F3_sim1_data = LTSpice.plot_data_polar("src/simulations/export/04_1 Aktiver RLC-Filter.txt")
    bode_F3_sim1_data_mag, bode_F3_sim1_data_pha = bode_F3_sim1_data["V(out)"]

    bode_F3_sim1_plot_mag = list_plot(
        bode_F3_sim1_data_mag,
        axes_labels=["Frequenz in Hz", "Betrag in dB"],
        scale='semilogx',
        figsize=[5.5,2],
        plotjoined=True,
        frame=True,
        color='darkcyan'
    )

    bode_F3_sim1_plot_pha = list_plot(
        bode_F3_sim1_data_pha,
        axes_labels=["Frequenz in Hz", "Phase in $\degree$"],
        scale='semilogx',
        figsize=[5.5,2],
        plotjoined=True,
        frame=True,
        color='darkcyan'
    )
\end{sagesilent}

\begin{figure}[H]
    \centering
    \begin{subfigure}{\textwidth}
        \centering
        \sageplot{bode_F3_sim1_plot_mag}
    \end{subfigure}
    \quad
    \begin{subfigure}{\textwidth}
        \centering
        \sageplot{bode_F3_sim1_plot_pha}
    \end{subfigure}
    \caption{Amplituden- und Phasengang der Simulation des aktiven RLC-Filters}
    \label{fig:F3_Sim1}
\end{figure}

\subsection{Analyse}

Bei der Wahl des OPV muss auf das Gain-Bandwidth-Product (GBW) geachtet werden.
Es stellt eine obere Grenze für die maximale Frequenz dar, bei der der OPV noch eine annehmbare Verstärkung liefern kann.
Deshalb muss das GBW höher, als die höchste Frequenz sein, die im Filter verarbeitet werden.

Des weiteren sollten natürlich auch andere Faktoren wie die Verstärkungsstabilität, Rauschspannungen und -ströme sowie die Eingangsimpedanz berücksichtigt werden, um das beste Ergebnis zu erzielen.

\subsection{Gegenkopplung mit Spannungsteiler}

\begin{sagesilent}
    R_1 = 300
    R_2 = 600
\end{sagesilent}

\begin{figure}[H]
    \centering
    \begin{circuitikz}
        \coordinate (in+) at (0,0);
        \coordinate (in-) at (0,-4);

        \draw
        (in+) to[R=$R$, a=$\SIvar{R}{\ohm}$, o-] ++(2,0)
        to[C=$C$, a=$\SIvar{C_1*10^3, 1}{\milli\farad}$] ++(2,0) coordinate (aux1)
        -- ++(2,0) node[op amp,anchor=-](OPV){OPV}

        (OPV.out) to[short, -o] ++(1,0) coordinate (out+)

        (OPV.+) -- ++(0, -1) coordinate(opv-aux1)
        to[R=$R2$, a=$\SIvar{R_2}{\ohm}$] (opv-aux1 -| OPV.out) to[short,-*] (OPV.out)

        (opv-aux1) to[R=$R1$, a=$\SIvar{R_1}{\ohm}$,*-*] (opv-aux1 |- in-)
        (aux1) to[cute inductor=$L$, a=$\SIvar{L, 3}{\henry}$, *-*] (aux1 |- in-)

        (in-) to[short,o-o] (in- -| out+) coordinate (out-)

        (in+) to[open,name=inV,v] (in-)
        (out+) to[open,name=outV,v^] (out-)
        ;

        \voltage{inV}{$U_e$}
        \voltage{outV}{$U_a$}
    \end{circuitikz}
\end{figure}

\begin{equation*}
    F_{(jw)}
    = \frac{\underline{U_{a}}}{\underline{U_{e}}}
    = \frac{jwL}{jwL+R+\frac{1}{jwC}} \cdot (1+\frac{R_1}{R_2})
\end{equation*}

\begin{sagesilent}
    bode_F3 = TransferFunction((s*L)/(s*L+R+(1/(s*C_1)))*(1+(R_1/R_2)), 1, 10000, 1)
\end{sagesilent}

\subsubsection{Simulation}

\begin{sagesilent}
    bode_F3_sim2_data = LTSpice.plot_data_polar("src/simulations/export/04_2 Aktiver RLC-Filter Gegenkopplung.txt")
    bode_F3_sim2_data_mag, bode_F3_sim2_data_pha = bode_F3_sim2_data["V(out)"]

    bode_F3_sim2_plot_mag = list_plot(
        bode_F3_sim2_data_mag,
        axes_labels=["Frequenz in Hz", "Betrag in dB"],
        legend_label="mit Spannungsteiler $\it{(R/2R)}$",
        scale='semilogx',
        figsize=[5.5,2],
        plotjoined=True,
        frame=True,
        color='darkcyan'
    )

    bode_F3_sim2_plot_mag += list_plot(
        bode_F3_sim1_data_mag,
        legend_label="ohne Spannungsteiler",
        plotjoined=True,
        linestyle=':',
        color='darkslateblue'
    )

    bode_F3_sim2_plot_pha = list_plot(
        bode_F3_sim2_data_pha,
        axes_labels=["Frequenz in Hz", "Phase in $\degree$"],
        legend_label="mit Spannungsteiler $\it{(R/2R)}$",
        scale='semilogx',
        figsize=[5.5,2],
        plotjoined=True,
        frame=True,
        color='darkcyan'
    )

    bode_F3_sim2_plot_pha += list_plot(
        bode_F3_sim1_data_pha,
        legend_label="ohne Spannungsteiler",
        plotjoined=True,
        linestyle=':',
        color='darkslateblue'
    )
\end{sagesilent}

\begin{figure}[H]
    \centering
    \begin{subfigure}{\textwidth}
        \centering
        \sageplot{bode_F3_sim2_plot_mag}
    \end{subfigure}
    \quad
    \begin{subfigure}{\textwidth}
        \centering
        \sageplot{bode_F3_sim2_plot_pha}
    \end{subfigure}
    \caption{Amplituden- und Phasengang der Simulation des aktiven RLC-Filters mit einer $R/2R$ Spannunsteiler-Gegenkopplung}
    \label{fig:F3_Sim2}
\end{figure}

\subsubsection{Analyse}

Die zusätzliche Gegenkopplung bietet die Möglichkeit, Signale im Filter zu Verstärken sowie zu kontrollieren und regulieren.
Des weitere kann der Frequenzgang optimiert werden. 

Jedoch könnte eine zu starke Gegenkopplung zu einer Verzerrung des Signals führen.
Die zusätzlichen Bauteile könnten jedoch auch Phasenverschiebungen verursachen.

\subsection{Weitere Analyse}

\subsubsection{Sprungantwort}

\begin{sagesilent}
    bode_F3_step = bode_F3.plot_data_step()
    
    bode_F3_plot_step = list_plot(
        bode_F3_step,
        axes_labels=["Zeit in s", "Amplitude"],
        figsize=[5.5,2],
        plotjoined=True,
        frame=True,
        color='red'
    )
\end{sagesilent}

\begin{figure}[H]
    \centering
    \sageplot{bode_F3_plot_step}
    \caption{Sprungantwort des aktiven RLC-Filters mit einer $R/2R$ Spannunsteiler-Gegenkopplung.}
    \label{fig:F3_Step}
\end{figure}

\subsubsection{Gruppenlaufzeit}

\begin{sagesilent}
    # Group-Delay LaTeX Formular:
    # var('w R R_1 R_2 L C')
    # s = I*w
    # F3_F = (s*L)/(s*L+R+(1/(s*C)))*(1+(R_1/R_2))
    # F3_phi = arctan( (F3_F.real()) / (F3_F.imag()) ) 
    # derivative(F3_phi, w).full_simplify()
    
    w_f200 = 2*pi*200
    bode_F3_G_f200 = bode_F3.get_group_delay(w_f200)
\end{sagesilent}

\begin{align*}
    \varphi &= \arctan\left(\frac{C L w^{2} - 1}{C R w}\right) \\
    G &= \frac{C^{2} L R w^{2} + C R}{C^{2} L^{2} w^{4} + {\left(C^{2} R^{2} - 2 \, C L\right)} w^{2} + 1} \\
    G &= \SIvar{bode_F3_G_f200 * 10^3, 3}{\milli\second} \tag*{bei $f=\qty{200}{\hertz}$}
\end{align*}

\begin{sagesilent}
    bode_F3_G = bode_F3.plot_data_freq_gd(f_stop=250)

    bode_F3_plot_G = list_plot(
        bode_F3_G,
        axes_labels=["Frequenz in Hz", "Zeit in s"],
        figsize=[5.5,2],
        plotjoined=True,
        frame=True,
        color='red'
    )
\end{sagesilent}

\begin{figure}[H]
    \centering
    \sageplot{bode_F3_plot_G}
    \caption{Gruppenlaufzeit des aktiven RLC-Filters mit einer $R/2R$ Spannunsteiler-Gegenkopplung.}
    \label{fig:F3_G}
\end{figure}

\subsubsection{Simulation mit Belastung $R_L=100\Omega$}

\begin{sagesilent}
    bode_F3_sim3_data = LTSpice.plot_data_polar("src/simulations/export/04_3 Aktiver RLC-Filter Gegenkopplung Belastung.txt")
    bode_F3_sim3_data_mag, bode_F3_sim3_data_pha = bode_F3_sim3_data["V(out)"]

    bode_F3_sim3_plot_mag = list_plot(
        bode_F3_sim3_data_mag,
        axes_labels=["Frequenz in Hz", "Betrag in dB"],
        legend_label="mit $R_L=100\Omega$",
        scale='semilogx',
        figsize=[5.5,2],
        plotjoined=True,
        frame=True,
        color='darkcyan'
    )

    bode_F3_sim3_plot_mag += list_plot(
        bode_F3_sim2_data_mag,
        legend_label="ohne Belastung",
        plotjoined=True,
        linestyle=':',
        color='darkslateblue'
    )

    bode_F3_sim3_plot_pha = list_plot(
        bode_F3_sim3_data_pha,
        axes_labels=["Frequenz in Hz", "Phase in $\degree$"],
        legend_label="mit $R_L=100\Omega$",
        scale='semilogx',
        figsize=[5.5,2],
        plotjoined=True,
        frame=True,
        color='darkcyan'
    )

    bode_F3_sim3_plot_pha += list_plot(
        bode_F3_sim2_data_pha,
        legend_label="ohne Belastung",
        plotjoined=True,
        linestyle=':',
        color='darkslateblue'
    )
\end{sagesilent}

\begin{figure}[H]
    \centering
    \begin{subfigure}{\textwidth}
        \centering
        \sageplot{bode_F3_sim3_plot_mag}
    \end{subfigure}
    \quad
    \begin{subfigure}{\textwidth}
        \centering
        \sageplot{bode_F3_sim3_plot_pha}
    \end{subfigure}
    \caption{Amplituden- und Phasengang der Simulation des aktiven RLC-Filters mit einer $R/2R$ Spannunsteiler-Gegenkopplung unter Belastung}
    \label{fig:F3_Sim3}
\end{figure}


\vfill

\IncludeHistoryTimeline

\end{document}