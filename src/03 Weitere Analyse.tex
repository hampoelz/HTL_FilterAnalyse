\section{Weitere Analyse von Filter 1 und 2}

\subsection{Sprungantwort}

\begin{sagesilent}
    bode_F1_step = bode_F1.plot_data_step()
    bode_F2_step = bode_F2_C1.plot_data_step()
    
    bode_F12_plot_step = list_plot(
        bode_F1_step,
        axes_labels=["Zeit in s", "Amplitude"],
        legend_label="RL-Filter",
        figsize=[5.5,2],
        plotjoined=True,
        frame=True,
        color='green'
    )

    bode_F12_plot_step += list_plot(
        bode_F2_step,
        legend_label="RLC-Filter",
        plotjoined=True,
        color='blue'
    )
\end{sagesilent}

\begin{figure}[H]
    \centering
    \sageplot{bode_F12_plot_step}
    \caption{Sprungantwort des RL- und RLC-Hochpasses.}
    \label{fig:F12_Step}
\end{figure}

Die Sprungantwort beschreibt, wie sich das Ausgangssignal ändert, wenn ein plötzlicher Sprung im Eingangssignal auftritt.
Dieser Sprung kann zum Beispiel durch einen schnellen Übergang von einem niedrigen auf eine hohes Eingangssignal verursacht werden.

\subsection{Gruppenlaufzeit}

Die Gruppenlaufzeit kann aus dem Verlauf der Phase berechnet werden. Sie ist die negative Ableitung des Phasenverlaufs nach der Kreisfrequenz. Für die Gruppenlaufzeit $G$ ergibt sich also:

\begin{equation*}
    G = \frac{d \varphi}{d w}
\end{equation*}

Um eine Ableitung berechnen zu können muss der allgemeine Phasengang $\varphi$ vorliegen. Es folgt dann die Ableitung nach $w$:

\vspace{1em}

\begin{sagesilent}
    # Group-Delay LaTeX Formular:
    # var('w R L C')
    # phi = arctan(-(R/(w*L)))
    # phi = arctan((1-(1/(w^2*L*C)))/(R/(w*L)))
    # latex(derivative(phi, w).full_simplify())

    w_f50 = 2*pi*50
    bode_F1_G_f50 = bode_F1.get_group_delay(w_f50)
    bode_F2_G_f50 = bode_F2.get_group_delay(w_f50)
\end{sagesilent}

\textbf{RL-Filter:}

\begin{align*}
    \varphi &= arctan(-\frac{R}{wL}) \\
    G &= \frac{R l}{l^{2} w^{2} + R^{2}} \\
    G &= \SIvar{bode_F1_G_f50 * 10^3, 3}{\milli\second} \tag*{bei $f=\qty{50}{\hertz}$}
\end{align*}

\textbf{RLC-Filter:}

\begin{align*}
    \varphi &= arctan(\frac{1 - \frac{1}{w^2LC}}{\frac{R}{wL}}) \\
    G &= \frac{C^{2} L R w^{2} + C R}{C^{2} L^{2} w^{4} + {\left(C^{2} R^{2} - 2 \, C L\right)} w^{2} + 1} \\
    G &= \SIvar{bode_F2_G_f50 * 10^3, 3}{\milli\second} \tag*{bei $f=\qty{50}{\hertz}$}
\end{align*}

\vspace{1em}

\begin{sagesilent}
    bode_F1_G = bode_F1.plot_data_freq_gd(f_stop=250)
    bode_F2_G = bode_F2.plot_data_freq_gd(f_stop=250)

    bode_F12_plot_G = list_plot(
        bode_F1_G,
        axes_labels=["Frequenz in Hz", "Zeit in s"],
        legend_label="RL-Filter",
        figsize=[5.5,2],
        plotjoined=True,
        frame=True,
        color='green'
    )

    bode_F12_plot_G += list_plot(
        bode_F2_G,
        legend_label="RLC-Filter",
        plotjoined=True,
        color='blue'
    )
\end{sagesilent}

\begin{figure}[H]
    \centering
    \sageplot{bode_F12_plot_G}
    \caption{Gruppenlaufzeit des RL- und RLC-Hochpasses.}
    \label{fig:F12_G}
\end{figure}

Die Gruppenlaufzeit ist die Verzögerung eines Signals, beim Durchgang durch einen Filter.
Sie beschreibt, wie lange eine Signalwelle durch einen Filter benötigt.

\subsection{Simulation mit Belastung $R_L=100\Omega$}

\begin{sagesilent}
    bode_F12_sim_data = LTSpice.plot_data_polar("src/simulations/export/03 RL- RLC-Filter Belastung.txt")

    bode_F12_sim_data_F1_mag, bode_F12_sim_data_F1_pha = bode_F12_sim_data["V(rl_out)"]
    bode_F12_sim_data_F2_mag, bode_F12_sim_data_F2_pha = bode_F12_sim_data["V(rlc_out)"]

    bode_F12_sim_plot_mag = list_plot(
        bode_F12_sim_data_F1_mag,
        axes_labels=["Frequenz in Hz", "Betrag in dB"],
        legend_label="RL-Filter",
        scale='semilogx',
        figsize=[5.5,2],
        plotjoined=True,
        frame=True,
        color='darkcyan'
    )

    bode_F12_sim_plot_mag += list_plot(
        bode_F12_sim_data_F2_mag,
        legend_label="RLC-Filter",
        plotjoined=True,
        color='darkslateblue'
    )

    bode_F12_sim_plot_pha = list_plot(
        bode_F12_sim_data_F1_pha,
        axes_labels=["Frequenz in Hz", "Phase in $\degree$"],
        legend_label="RL-Filter",
        scale='semilogx',
        figsize=[5.5,2],
        plotjoined=True,
        frame=True,
        color='darkcyan'
    )

    bode_F12_sim_plot_pha += list_plot(
        bode_F12_sim_data_F2_pha,
        legend_label="RLC-Filter",
        plotjoined=True,
        color='darkslateblue'
    )
\end{sagesilent}

\begin{figure}[H]
    \centering
    \begin{subfigure}{\textwidth}
        \centering
        \sageplot{bode_F12_sim_plot_mag}
    \end{subfigure}
    \quad
    \begin{subfigure}{\textwidth}
        \centering
        \sageplot{bode_F12_sim_plot_pha}
    \end{subfigure}
    \caption{Amplituden- und Phasengang der Simulation des RL- und RLC-Hochpasses unter Belastung}
    \label{fig:F12_Sim}
\end{figure}

Eine ohmsche Belastung beeinflusst die Koeffizienten, welche für die Berechnung der Übertragungsfunktion verwendet werden.
Dies führt dazu, dass die Grenzfrequenz des Filters sowie dessen Amplituden-Frequenzgang verändert werden.  