\section{Entwurf eines RLC-Filter zweiter Ordnung}

Tiefere Frequenzen sollen stärker abgeschwächt werden \textit{($\num{-40}\,\nicefrac{\unit{dB}}{\text{Dekade}}$)}.
Jedoch darf, zusätzlich zu den bereits verwendeten Bauelementen, nur eine Kapazität verwendet werden.

\begin{figure}[H]
    \centering
    \begin{circuitikz}
        \coordinate (in+) at (0,0);
        \coordinate (in-) at (0,-2);

        \draw
        (in+) to[R=$R$, a=$\SIvar{R}{\ohm}$, o-] ++(2,0)
        to[C=$C$] ++(2,0) coordinate (aux1)
        to[short,-o] ++(2,0) coordinate (out+)

        (aux1) to[cute inductor=$L$,*-*] (aux1 |- in-)

        (in-) to[short,o-o] (in- -| out+) coordinate (out-)

        (in+) to[open,name=inV,v] (in-)
        (out+) to[open,name=outV,v^] (out-)
        ;

        \voltage{inV}{$U_e$}
        \voltage{outV}{$U_a$}
    \end{circuitikz}
\end{figure}

\begin{equation*}
    F_{(jw)}
    = \frac{\underline{U_{a}}}{\underline{U_{e}}}
    = \frac{jwL}{jwL+R+\frac{1}{jwC}} \cdot \frac{\frac{1}{jwL}}{\frac{1}{jwL}}
    = \frac{1}{1+\frac{\frac{1}{jwC}}{jwL}+\frac{R}{jwL}}
    = \frac{1}{\underbrace{1-\frac{1}{w^2LC}}_{Re} + \underbrace{\frac{R}{jwL}}_{Im}}
\end{equation*}

\subsection{Betrag und Phase aus der Übertragungsfunktion}

\begin{align*}
    Re &= 1 - \frac{1}{w^2LC} \\
    Im &= \frac{R}{jwL}
\end{align*}

\begin{align*}
    |G_{jw}| & = \frac{1}{\sqrt{(1 - \frac{1}{w^2LC})^2 + (\frac{R}{wL})^2}} &
    \varphi &= arctan(\frac{1 - \frac{1}{w^2LC}}{\frac{R}{wL}})
\end{align*}

\subsection{Berechnung der Bauteile}

\begin{sagesilent}
    L = R/(2*pi*f_g)
    C = var('C')

    X_L = 2*pi*f_g*L
    X_C = 1/(2*pi*f_g*C)

    C = solve([X_L == X_C], C)[0].rhs()
\end{sagesilent}

\begin{align*}
    f_{g} &= \frac{R}{2 \pi L} \\
    \Rightarrow L &= \frac{R}{2 \pi f_{g}} \\
    L &= \frac{\var{R}}{2 \pi \var{f_g}} \\
    L &= \SIvar{L, 3}{\henry}
\end{align*}

\begin{align*}
    X_{L} &= 2 \pi f_{g} L \\
    X_{C} &= \frac{1}{2 \pi f_{g} C} \\
    X_{L} &= X_{C} \\
    \Rightarrow C &= \SIvar{C*10^6, 1}{\micro\farad}
\end{align*}

\subsection{Übertragungsfunktion}

\begin{sagesilent}
    bode_F2 = TransferFunction((s*L)/(s*L+R+1/(s*C)), 1, 10000, 1)
    bode_F2_freq_mag = bode_F2.plot_data_freq_mag_bode()
    bode_F2_freq_pha = bode_F2.plot_data_freq_pha_bode()

    bode_F2_plot_mag = list_plot(
        bode_F2_freq_mag,
        axes_labels=["Frequenz in Hz", "Betrag in dB"],
        legend_label="RLC-Filter",
        scale='semilogx',
        figsize=[5.5,2],
        plotjoined=True,
        frame=True,
        color='red'
    )

    bode_F2_plot_mag += list_plot(
        bode_F1_freq_mag,
        legend_label="RL-Filter",
        plotjoined=True,
        linestyle=':',
        color='blue'
    )

    bode_F2_plot_pha = list_plot(
        bode_F2_freq_pha,
        axes_labels=["Frequenz in Hz", "Phase in $\degree$"],
        legend_label="RLC-Filter",
        scale='semilogx',
        figsize=[5.5,2],
        plotjoined=True,
        frame=True,
        color='red'
    )

    bode_F2_plot_pha += list_plot(
        bode_F1_freq_pha,
        legend_label="RL-Filter",
        plotjoined=True,
        linestyle=':',
        color='blue'
    )
\end{sagesilent}

\begin{figure}[H]
    \centering
    \begin{subfigure}{\textwidth}
        \centering
        \sageplot{bode_F2_plot_mag}
    \end{subfigure}
    \quad
    \begin{subfigure}{\textwidth}
        \centering
        \sageplot{bode_F2_plot_pha}
    \end{subfigure}
    \caption{Amplituden- und Phasengang des RLC-Hochpasses und Vergleich mit dem RL-Hochpass (Abb. \ref{fig:F1}) der vorherigen Übung.}
    \label{fig:F2}
\end{figure}

\subsection{Übertragungsfunktion mit $C_1=0.1mF$ und $C_2=0.01mF$}

\begin{sagesilent}
    C_1 = 0.1*10^-3
    bode_F2_C1 = TransferFunction((s*L)/(s*L+R+1/(s*C_1)), 1, 10000, 1)
    bode_F2_C1_freq_mag = bode_F2_C1.plot_data_freq_mag_bode()
    bode_F2_C1_freq_pha = bode_F2_C1.plot_data_freq_pha_bode()

    C_2 = 0.01*10^-3
    bode_F2_C2 = TransferFunction((s*L)/(s*L+R+1/(s*C_2)), 1, 10000, 1)
    bode_F2_C2_freq_mag = bode_F2_C2.plot_data_freq_mag_bode()
    bode_F2_C2_freq_pha = bode_F2_C2.plot_data_freq_pha_bode()
    
    bode_F2_C12_plot_mag = list_plot(
        bode_F2_C1_freq_mag,
        axes_labels=["Frequenz in Hz", "Betrag in dB"],
        legend_label="$C_1$",
        scale='semilogx',
        figsize=[5.5,2],
        plotjoined=True,
        frame=True,
        color='green'
    )

    bode_F2_C12_plot_mag += list_plot(
        bode_F2_C2_freq_mag,
        legend_label='$C_2$',
        plotjoined=True,
        color='purple'
    )

    bode_F2_C12_plot_pha = list_plot(
        bode_F2_C1_freq_pha,
        axes_labels=["Frequenz in Hz", "Phase in $\degree$"],
        legend_label="$C_1$",
        scale='semilogx',
        figsize=[5.5,2],
        plotjoined=True,
        frame=True,
        color='green'
    )

    bode_F2_C12_plot_pha += list_plot(
        bode_F2_C2_freq_pha,
        legend_label="$C_2$",
        plotjoined=True,
        color='purple'
    )
\end{sagesilent}

\begin{figure}[H]
    \centering
    \begin{subfigure}{\textwidth}
        \centering
        \sageplot{bode_F2_C12_plot_mag}
    \end{subfigure}
    \quad
    \begin{subfigure}{\textwidth}
        \centering
        \sageplot{bode_F2_C12_plot_pha}
    \end{subfigure}
    \caption{Amplituden- und Phasengang des RLC-Hochpasses.}
    \label{fig:F2_C12}
\end{figure}

\subsection{Analyse}

Die Kapazität $C$ hat einen direkten Einfluss auf die Übertragungsbandbreite des RLC-Hochpassfilters und bestimmt in gewisser Weise, wie gut der Filter bestimmte Frequenzen durchlässt oder abfiltert.
Bei größeren Kapazitäten, lässt der der Hochpassfilter eine größere Bandbreite an höheren Frequenzen durch.

Bei der Kapazität $C_2$ (\SIvar{C_1*10^3, 1}{\milli\farad}) stimmt die Grenzfrequenz $f_g$ von \SIvar{f_g}{\hertz} nicht mehr überein und ist daher nicht geeignet, deshalb wird für weitere Analysen die Kapazität $C_2$ (\SIvar{C_2*10^3, 2}{\milli\farad}) verwendet.

\begin{sagesilent}
    bode_F2_mag_1 = bode_F2_C1_freq_mag[0][1]
    bode_F2_mag_10 = bode_F2_C1_freq_mag[9][1]

    attenuation_dB_F2 = bode_F2_mag_10 - bode_F2_mag_1
    attenuation_dB_F2 = attenuation_dB_F2
\end{sagesilent}

Die Abschwächung im Frequenzbereich von \qty{1}{\hertz} bis \qty{10}{\hertz} beträgt somit $\var{attenuation_dB_F2, 0}\,\nicefrac{\unit{dB}}{\text{Dekade}}$.

\begin{sagesilent}
    bode_F2_mag_fg = bode_F2_C1_freq_mag[f_g-1][1]
    bode_F2_pha_fg = bode_F2_C1_freq_pha[f_g-1][1]
\end{sagesilent}

Bei der Grenzfrequenz von \SIvar{f_g}{\hertz} ergibt sich dabei eine Verstärkung von \SIvar{bode_F2_mag_fg, 2}{\decibel} und eine Phasenverschiebung von \SIvar{bode_F2_pha_fg, 0}{\degree}

\subsection{Simulation}

\begin{sagesilent}
    bode_F2_sim_data = LTSpice.plot_data_polar("src/simulations/export/02 RLC-Filter.txt")
    bode_F2_sim_data_mag, bode_F1_sim_data_pha = bode_F2_sim_data["V(out)"]

    bode_F2_sim_plot_mag = list_plot(
        bode_F2_sim_data_mag,
        axes_labels=["Frequenz in Hz", "Betrag in dB"],
        scale='semilogx',
        figsize=[5.5,2],
        plotjoined=True,
        frame=True,
        color='darkcyan'
    )

    bode_F2_sim_plot_pha = list_plot(
        bode_F1_sim_data_pha,
        axes_labels=["Frequenz in Hz", "Phase in $\degree$"],
        scale='semilogx',
        figsize=[5.5,2],
        plotjoined=True,
        frame=True,
        color='darkcyan'
    )
\end{sagesilent}

\begin{figure}[H]
    \centering
    \begin{subfigure}{\textwidth}
        \centering
        \sageplot{bode_F2_sim_plot_mag}
    \end{subfigure}
    \quad
    \begin{subfigure}{\textwidth}
        \centering
        \sageplot{bode_F2_sim_plot_pha}
    \end{subfigure}
    \caption{Amplituden- und Phasengang der Simulation des RLC-Hochpasses}
    \label{fig:F2_Sim}
\end{figure}