\section{Entwurf eines aktiven RLC-Filter zweiter Ordnung}

Im Zusammenhang mit den ermittelten Bauteilen vom RLC-Filter, soll ein aktiver Filter mit einem Operationsverstärker OP747 entworfen werden.
Der OPV soll als Spannungsfolger im Dual-Supply Mode eingesetzt werden.

\begin{figure}[H]
    \centering
    \begin{circuitikz}
        \coordinate (in+) at (0,0);
        \coordinate (in-) at (0,-3);

        \draw
        (in+) to[R=$R$, a=$\SIvar{R}{\ohm}$, o-] ++(2,0)
        to[C=$C$, a=$\SIvar{C_1*10^3, 1}{\milli\farad}$] ++(2,0) coordinate (aux1)
        -- ++(2,0) node[op amp,anchor=-](OPV){OPV}

        (OPV.out) to[short, -o] ++(1,0) coordinate (out+)

        (OPV.+) -- ++(0, -1) coordinate(opv-aux1)
        -- (opv-aux1 -| OPV.out) to[short,-*] (OPV.out)
        
        (aux1) to[cute inductor=$L$, a=$\SIvar{L, 3}{\henry}$, *-*] (aux1 |- in-)

        (in-) to[short,o-o] (in- -| out+) coordinate (out-)

        (in+) to[open,name=inV,v] (in-)
        (out+) to[open,name=outV,v^] (out-)
        ;

        \voltage{inV}{$U_e$}
        \voltage{outV}{$U_a$}
    \end{circuitikz}
\end{figure}

\begin{equation*}
    F_{(jw)}
    = \frac{\underline{U_{a}}}{\underline{U_{e}}}
    = \frac{jwL}{jwL+R+\frac{1}{jwC}} \cdot \frac{\frac{1}{jwL}}{\frac{1}{jwL}}
    = \frac{1}{1+\frac{\frac{1}{jwC}}{jwL}+\frac{R}{jwL}}
    = \frac{1}{\underbrace{1-\frac{1}{w^2LC}}_{Re} + \underbrace{\frac{R}{jwL}}_{Im}}
\end{equation*}

\subsection{Betrag und Phase aus der Übertragungsfunktion}

\begin{align*}
    Re &= 1 - \frac{1}{w^2LC} \\
    Im &= \frac{R}{jwL}
\end{align*}

\begin{align*}
    |G_{jw}| & = \frac{1}{\sqrt{(1 - \frac{1}{w^2LC})^2 + (\frac{R}{wL})^2}} &
    \varphi &= arctan(\frac{1 - \frac{1}{w^2LC}}{\frac{R}{wL}})
\end{align*}

\subsection{Simulation}

\begin{sagesilent}
    bode_F3_sim1_data = LTSpice.plot_data_polar("src/simulations/export/04_1 Aktiver RLC-Filter.txt")
    bode_F3_sim1_data_mag, bode_F3_sim1_data_pha = bode_F3_sim1_data["V(out)"]

    bode_F3_sim1_plot_mag = list_plot(
        bode_F3_sim1_data_mag,
        axes_labels=["Frequenz in Hz", "Betrag in dB"],
        scale='semilogx',
        figsize=[5.5,2],
        plotjoined=True,
        frame=True,
        color='darkcyan'
    )

    bode_F3_sim1_plot_pha = list_plot(
        bode_F3_sim1_data_pha,
        axes_labels=["Frequenz in Hz", "Phase in $\degree$"],
        scale='semilogx',
        figsize=[5.5,2],
        plotjoined=True,
        frame=True,
        color='darkcyan'
    )
\end{sagesilent}

\begin{figure}[H]
    \centering
    \begin{subfigure}{\textwidth}
        \centering
        \sageplot{bode_F3_sim1_plot_mag}
    \end{subfigure}
    \quad
    \begin{subfigure}{\textwidth}
        \centering
        \sageplot{bode_F3_sim1_plot_pha}
    \end{subfigure}
    \caption{Amplituden- und Phasengang der Simulation des aktiven RLC-Filters}
    \label{fig:F3_Sim1}
\end{figure}

\subsection{Analyse}

Bei der Wahl des OPV muss auf das Gain-Bandwidth-Product (GBW) geachtet werden.
Es stellt eine obere Grenze für die maximale Frequenz dar, bei der der OPV noch eine annehmbare Verstärkung liefern kann.
Deshalb muss das GBW höher, als die höchste Frequenz sein, die im Filter verarbeitet werden.

Des weiteren sollten natürlich auch andere Faktoren wie die Verstärkungsstabilität, Rauschspannungen und -ströme sowie die Eingangsimpedanz berücksichtigt werden, um das beste Ergebnis zu erzielen.

\subsection{Gegenkopplung mit Spannungsteiler}

\begin{sagesilent}
    R_1 = 300
    R_2 = 600
\end{sagesilent}

\begin{figure}[H]
    \centering
    \begin{circuitikz}
        \coordinate (in+) at (0,0);
        \coordinate (in-) at (0,-4);

        \draw
        (in+) to[R=$R$, a=$\SIvar{R}{\ohm}$, o-] ++(2,0)
        to[C=$C$, a=$\SIvar{C_1*10^3, 1}{\milli\farad}$] ++(2,0) coordinate (aux1)
        -- ++(2,0) node[op amp,anchor=-](OPV){OPV}

        (OPV.out) to[short, -o] ++(1,0) coordinate (out+)

        (OPV.+) -- ++(0, -1) coordinate(opv-aux1)
        to[R=$R2$, a=$\SIvar{R_2}{\ohm}$] (opv-aux1 -| OPV.out) to[short,-*] (OPV.out)

        (opv-aux1) to[R=$R1$, a=$\SIvar{R_1}{\ohm}$,*-*] (opv-aux1 |- in-)
        (aux1) to[cute inductor=$L$, a=$\SIvar{L, 3}{\henry}$, *-*] (aux1 |- in-)

        (in-) to[short,o-o] (in- -| out+) coordinate (out-)

        (in+) to[open,name=inV,v] (in-)
        (out+) to[open,name=outV,v^] (out-)
        ;

        \voltage{inV}{$U_e$}
        \voltage{outV}{$U_a$}
    \end{circuitikz}
\end{figure}

\begin{equation*}
    F_{(jw)}
    = \frac{\underline{U_{a}}}{\underline{U_{e}}}
    = \frac{jwL}{jwL+R+\frac{1}{jwC}} \cdot (1+\frac{R_1}{R_2})
\end{equation*}

\begin{sagesilent}
    bode_F3 = TransferFunction((s*L)/(s*L+R+(1/(s*C_1)))*(1+(R_1/R_2)), 1, 10000, 1)
\end{sagesilent}

\subsubsection{Simulation}

\begin{sagesilent}
    bode_F3_sim2_data = LTSpice.plot_data_polar("src/simulations/export/04_2 Aktiver RLC-Filter Gegenkopplung.txt")
    bode_F3_sim2_data_mag, bode_F3_sim2_data_pha = bode_F3_sim2_data["V(out)"]

    bode_F3_sim2_plot_mag = list_plot(
        bode_F3_sim2_data_mag,
        axes_labels=["Frequenz in Hz", "Betrag in dB"],
        legend_label="mit Spannungsteiler $\it{(R/2R)}$",
        scale='semilogx',
        figsize=[5.5,2],
        plotjoined=True,
        frame=True,
        color='darkcyan'
    )

    bode_F3_sim2_plot_mag += list_plot(
        bode_F3_sim1_data_mag,
        legend_label="ohne Spannungsteiler",
        plotjoined=True,
        linestyle=':',
        color='darkslateblue'
    )

    bode_F3_sim2_plot_pha = list_plot(
        bode_F3_sim2_data_pha,
        axes_labels=["Frequenz in Hz", "Phase in $\degree$"],
        legend_label="mit Spannungsteiler $\it{(R/2R)}$",
        scale='semilogx',
        figsize=[5.5,2],
        plotjoined=True,
        frame=True,
        color='darkcyan'
    )

    bode_F3_sim2_plot_pha += list_plot(
        bode_F3_sim1_data_pha,
        legend_label="ohne Spannungsteiler",
        plotjoined=True,
        linestyle=':',
        color='darkslateblue'
    )
\end{sagesilent}

\begin{figure}[H]
    \centering
    \begin{subfigure}{\textwidth}
        \centering
        \sageplot{bode_F3_sim2_plot_mag}
    \end{subfigure}
    \quad
    \begin{subfigure}{\textwidth}
        \centering
        \sageplot{bode_F3_sim2_plot_pha}
    \end{subfigure}
    \caption{Amplituden- und Phasengang der Simulation des aktiven RLC-Filters mit einer $R/2R$ Spannunsteiler-Gegenkopplung}
    \label{fig:F3_Sim2}
\end{figure}

\subsubsection{Analyse}

Die zusätzliche Gegenkopplung bietet die Möglichkeit, Signale im Filter zu Verstärken sowie zu kontrollieren und regulieren.
Des weitere kann der Frequenzgang optimiert werden. 

Jedoch könnte eine zu starke Gegenkopplung zu einer Verzerrung des Signals führen.
Die zusätzlichen Bauteile könnten jedoch auch Phasenverschiebungen verursachen.

\subsection{Weitere Analyse}

\subsubsection{Sprungantwort}

\begin{sagesilent}
    bode_F3_step = bode_F3.plot_data_step()
    
    bode_F3_plot_step = list_plot(
        bode_F3_step,
        axes_labels=["Zeit in s", "Amplitude"],
        figsize=[5.5,2],
        plotjoined=True,
        frame=True,
        color='red'
    )
\end{sagesilent}

\begin{figure}[H]
    \centering
    \sageplot{bode_F3_plot_step}
    \caption{Sprungantwort des aktiven RLC-Filters mit einer $R/2R$ Spannunsteiler-Gegenkopplung.}
    \label{fig:F3_Step}
\end{figure}

\subsubsection{Gruppenlaufzeit}

\begin{sagesilent}
    # Group-Delay LaTeX Formular:
    # var('w R R_1 R_2 L C')
    # s = I*w
    # F3_F = (s*L)/(s*L+R+(1/(s*C)))*(1+(R_1/R_2))
    # F3_phi = arctan( (F3_F.real()) / (F3_F.imag()) ) 
    # derivative(F3_phi, w).full_simplify()
    
    w_f200 = 2*pi*200
    bode_F3_G_f200 = bode_F3.get_group_delay(w_f200)
\end{sagesilent}

\begin{align*}
    \varphi &= \arctan\left(\frac{C L w^{2} - 1}{C R w}\right) \\
    G &= \frac{C^{2} L R w^{2} + C R}{C^{2} L^{2} w^{4} + {\left(C^{2} R^{2} - 2 \, C L\right)} w^{2} + 1} \\
    G &= \SIvar{bode_F3_G_f200 * 10^3, 3}{\milli\second} \tag*{bei $f=\qty{200}{\hertz}$}
\end{align*}

\begin{sagesilent}
    bode_F3_G = bode_F3.plot_data_freq_gd(f_stop=250)

    bode_F3_plot_G = list_plot(
        bode_F3_G,
        axes_labels=["Frequenz in Hz", "Zeit in s"],
        figsize=[5.5,2],
        plotjoined=True,
        frame=True,
        color='red'
    )
\end{sagesilent}

\begin{figure}[H]
    \centering
    \sageplot{bode_F3_plot_G}
    \caption{Gruppenlaufzeit des aktiven RLC-Filters mit einer $R/2R$ Spannunsteiler-Gegenkopplung.}
    \label{fig:F3_G}
\end{figure}

\subsubsection{Simulation mit Belastung $R_L=100\Omega$}

\begin{sagesilent}
    bode_F3_sim3_data = LTSpice.plot_data_polar("src/simulations/export/04_3 Aktiver RLC-Filter Gegenkopplung Belastung.txt")
    bode_F3_sim3_data_mag, bode_F3_sim3_data_pha = bode_F3_sim3_data["V(out)"]

    bode_F3_sim3_plot_mag = list_plot(
        bode_F3_sim3_data_mag,
        axes_labels=["Frequenz in Hz", "Betrag in dB"],
        legend_label="mit $R_L=100\Omega$",
        scale='semilogx',
        figsize=[5.5,2],
        plotjoined=True,
        frame=True,
        color='darkcyan'
    )

    bode_F3_sim3_plot_mag += list_plot(
        bode_F3_sim2_data_mag,
        legend_label="ohne Belastung",
        plotjoined=True,
        linestyle=':',
        color='darkslateblue'
    )

    bode_F3_sim3_plot_pha = list_plot(
        bode_F3_sim3_data_pha,
        axes_labels=["Frequenz in Hz", "Phase in $\degree$"],
        legend_label="mit $R_L=100\Omega$",
        scale='semilogx',
        figsize=[5.5,2],
        plotjoined=True,
        frame=True,
        color='darkcyan'
    )

    bode_F3_sim3_plot_pha += list_plot(
        bode_F3_sim2_data_pha,
        legend_label="ohne Belastung",
        plotjoined=True,
        linestyle=':',
        color='darkslateblue'
    )
\end{sagesilent}

\begin{figure}[H]
    \centering
    \begin{subfigure}{\textwidth}
        \centering
        \sageplot{bode_F3_sim3_plot_mag}
    \end{subfigure}
    \quad
    \begin{subfigure}{\textwidth}
        \centering
        \sageplot{bode_F3_sim3_plot_pha}
    \end{subfigure}
    \caption{Amplituden- und Phasengang der Simulation des aktiven RLC-Filters mit einer $R/2R$ Spannunsteiler-Gegenkopplung unter Belastung}
    \label{fig:F3_Sim3}
\end{figure}
